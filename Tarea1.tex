\documentclass{article}
\usepackage{graphicx} % Required for inserting images
\usepackage{amsmath}
\usepackage{amssymb}
\usepackage[spanish]{babel}
\usepackage[utf8]{inputenc}
\usepackage{array}


\usepackage{listings} %Soporte para código


\title{Práctica Latex}
\author{Héctor Jerome Treviño Puebla, Gontran Villalobos, Joaquin Ramirez}
\date{August 2024}

\begin{document}
\section*{Un balón de futbol americano}
En este ejercicio se trata de calcular aproximadamente el volumen V de un balón
de football americano profesional suponiendo que su superficie se genera rotando,
alrededor del eje horizontal, una curva y = y (x) que satisface que la longitud del eje
entre las dos puntas del balón es de 28 cm y la circunferencia de la máxima sección
circular transversal al eje mide 53 cm (véase la siguiente figura).

a) Rebane la mitad derecha del balón en 10 cilindros de grosor $\Delta x = 1.4$ cm, de manera
que las rebanadas intersecten el eje x en los valores de la siguiente particiónn del intervalo
$[0, 14]$ 
1) Muestre que la parábola es, aproximadamente, la gráfica de la función
    $$y(x)=-0.043x^2 + 8.4352$$
definida para $-14\leq x\leq 14$


\begin{itemize}
    \item Sabemos que la $\text{Circunferencia}=2\pi r \therefore r=\frac{53}{2\pi}$
    \item la fórmula de las cónicas para la parábola es $y(x)=ax{^2}+bx+c$
    \item sabemos que el radio de la circunferencia del balón es la longitud del radio cuando $x=0$, queremos saber el valor de y para $x=0$
\end{itemize}

\begin{gather*}
\therefore y(0)=\frac{53}{2\pi}=a(0){^2}+b(0)+c=c\\
\Rightarrow c=\frac{53}{2\pi}
\end{gather*}
wTambién conocemos dos puntos que pasan por la ecuación del balón y son los puntos extremos, para encontrar los valores de $a,b$ sustituimos estos valores y resolvemos el sistema de ecuaciones
\begin{gather*}
y(14)=a(14){^2}+b(14)+c\\
y(-14)=a(-14){^2}+b(-14)+c\\
\text{-------------------------------------------------}\\
196a+c=0\\
+196a+c=0\\
\text{----------------------}\\
392a+2c\Rightarrow a=-\frac{2c}{392}=-\frac{c}{196}\\
\therefore y(x)=-\frac{c}{196}x{^2}+c\\
\Rightarrow y(x)=-\frac{\frac{53}{2\pi}}{196}x{^2}+\frac{53}{2\pi}\\
\end{gather*}
Simplificando tenemos que $$y(x)\approx-0.0430x{^2}+8.4352$$
Podemos expresar la función en términos de $r_0$, de esta manera podemos deducir el dominio de la función.
\begin{gather*}
y(x)=-\frac{c}{196}x{^2}+c\\
y(x)=-\frac{r_{o}}{196}x{^2}+r_{o},\text{donde }r_o=c\\
\text{Igualamos a cero para determinar las raices}\\
-\frac{r_{o}}{196}x{^2}+r_{o}=0\\
-\frac{r_{o}}{196}x{^2}=-r_{o}\\
-r_{0}x{^2}=-r_{0}(196)\\
x{^2}=\frac{-r_{0}(196)}{-r_{0}}\\
x=\pm \sqrt{196}\\
x=\pm 14\\
\therefore-14\leq x\leq 14
\end{gather*}
estos puntos, como habíamos definido antes, son los puntos de corte del centro del balón, solo estamos definiendo, en este caso la mitad de arriba del balón para aproximar su volumen. Esto de ve reflejado con el coeficiente $a$ negativo ($-a$).

2) Muestre que la aproximación del volumen (7) con los 10 cilindros circunscritos de grosor $\Delta x = 1.4 cm$ y radios $y(x_{k})$ cm para $k = 0, 1, 2, . . . , 9$ es:
$$V\approx 2 \; x \;1825.5143=3651.0286 cm^3$$

La aproximación del volumen para el balón de futbol es: \begin{gather*}
    \sum_{k=0}^{9}V_{k}\\
    \text{donde: } V_{k}=\pi*r_{k}{^2}*\Delta h\\\Delta h=1.4,\text{la altura de cada cilindro}\\
    \end{gather*}
    el valor de correspondencia para cada $x_{k}=\Delta h*k$, la hayamos sustituyendo en la fórmula $r_{k}=y(x_{k})=y(x)\approx-0.0430x_{k}{^2}+8.4352$.
Decimos que$$\sum_{k=0}^{9}V_{k}=\sum_{k=0}^{9}\pi*r_{k}{^2}\Delta h$$
Podemos decir que $\pi$ y $\Delta h$ son siempre constantes en la suma, por lo tanto podemos factorizar:

\begin{gather*}
=\pi \Delta h\sum_{k=0}^{9}r_{k}\\
=\pi \Delta h\sum_{k=0}^{9}(-0.0430x_{k}{^2}+8.4352){^2}
\end{gather*}

\begin{table}[!hbt]
    \begin{center}
    \begin{tabular}{| c | c | c | c | c | }
    \hline
    K & $\Delta h \cdot j$ & $y(x_{k})$ & $y(x_{k})^{2}$ & $\Delta h*\pi*y(x_{k}^2)$ \\ \hline
    0 & 0 & 8.4352 & 71.15259904 & 312.9462072 \\
    1 & 1.4 & 8.35092 & 69.73786485 & 306.7238667 \\
    2 & 2.8 & 8.09808 & 65.57889969 & 288.4317398 \\
    3 & 4.2 & 7.67668 & 58.93141582 & 259.1945103 \\
    4 & 5.6 & 7.08672 & 50.22160036 & 220.8866516 \\
    5 & 7 & 6.3282 & 40.04611524 & 176.1324259  \\
    6 & 8.4 & 5.40112 & 29.17209725 & 128.305885 \\
    7 & 9.8 & 4.30548 & 18.53715803 & 81.53086994  \\
    8 & 11.2 & 3.04128 & 9.249384038 & 40.68101085  \\
    9 & 12.6 & 1.60852 & 2.58733659 & 11.37972729  \\ \hline
\multicolumn{5}{ |r| } {suma $ = 1825.5143\;cm{^3}$}\\
\multicolumn{5}{|r|}{$V \approx 2 x 1825.5143 = 3651.0286 cm3$}\\ \hline
    \end{tabular}
    \caption{Tabla de suma de los factores $x_k$}
    \label{tab:la suma de los cilindros circunscritos como parabola}
    \end{center}
    \end{table}

    Ya que solo calculamos los radios de la parte derecha del balón y considerando que es simétrico, el volumen de el lado derecho es el mismo que el del lado izquierdo, por lo tanto se multiplica por 2.
Muestre que la aproximación del volumen (7) con los 10 cilindros inscritos de
grosor $\Delta x = 1.4$ cm y radios $y (x_{k})$ cm para $k = 0, 1, . . . , 9$ es:
$$V\approx 2 \; x \;1825.5143=3651.0286 cm^3$$
La aproximación del volumen para el balón de futbol es: 
\begin{gather*}
    \sum_{k=0}^{9}V_{k}\\
    \text{donde: } V_{k}=\pi*r_{k}{^2}*\Delta h\\\Delta h=1.4,\text{la altura de cada cilindro}\\
\end{gather*}
el valor de correspondencia para cada $x_{k}=\Delta h*k$, la hayamos sustituyendo en la fórmula $r_{k}=y(x_{k})=y(x)\approx-0.0430x_{k}{^2}+8.4352$.
Decimos que$$\sum_{k=0}^{9}V_{k}=\sum_{k=0}^{9}\pi*r_{k}{^2}\Delta h$$
Podemos decir que $\pi$ y $\Delta h$ son siempre constantes en la suma, por lo tanto podemos factorizar:
\begin{gather*}
=\pi \Delta h\sum_{k=0}^{9}r_{k}\\
=\pi \Delta h\sum_{k=0}^{9}(-0.0430x_{k}{^2}+8.4352){^2}
\end{gather*}
\begin{table}[!hbt]
    \begin{center}
    \begin{tabular}{| c | c | c | c | c | }
    \hline
    K & $\Delta h \cdot j$ & $y(x_{k})$ & $y(x_{k})^{2}$ & $\Delta h*\pi*y(x_{k}^2)$ \\ \hline
    1 & 1.4 & 8.35092 & 69.73786485 & 306.7238667 \\
    2 & 2.8 & 8.09808 & 65.57889969 & 288.4317398 \\
    3 & 4.2 & 7.67668 & 58.93141582 & 259.1945103 \\
    4 & 5.6 & 7.08672 & 50.22160036 & 220.8866516 \\
    5 & 7   & 6.3282  & 40.04611524 & 176.1324259 \\
    6 & 8.4 & 5.40112 & 29.17209725 & 128.305885  \\
    7 & 9.8 & 4.30548 & 18.53715803 & 81.53086994 \\
    8 & 11.2& 3.04128 & 9.249384038 & 40.68101085 \\
    9 & 12.6& 1.60852 & 2.58733659  & 11.37972729 \\
   10 & 14  & 0.0072  & 5.184E-05   & 0.000228005 \\ \hline
\multicolumn{5}{ |r| } {suma $ = 1825.5143\;cm{^3}$}\\
\multicolumn{5}{|r|}{$V \approx 2 x 1825.5143 = 3651.0286 cm3$}\\ \hline
    \end{tabular}
    \caption{Tabla de suma de los factores $x_k$}
    \label{tab:la suma de los cilindros inscritos}
    \end{center}
    \end{table}

    b) Suponga ahora que la curva $y = y (x)$ es la mitad superior de una elipse con
    centro en el origen, semieje horizontal $a = 14$ cm (sobre el eje x) y semieje vertical
    $$b = 53 2\pi \approx 8.4352 cm (sobre el eje y).$$
    Partimos de la ecuación simétrica de la parábola.
    $$\frac{x{^2}}{a{^2}}+\frac{y{^2}}{b{^2}}=1$$Definimos los vértices en $v(0,14)$ y $v'(0,-14)$, por definición sabemos que la mitad de la longitud del semieje mayor es $a$, entonces a=14, $b$ es la mitad del semieje menor, el cual es el radio de nuestro balón, $\Rightarrow b=r_{0}=\frac{53}{2\pi}$
    $$\frac{x{^2}}{14{^2}}+\frac{y{^2}}{r_{0}{^2}}=1$$
    Reducimos hasta despejar a y.
    \begin{gather*}
    14{^2}r_{0}{^2}\left( \frac{x{^2}}{14{^2}} +\frac{y{^2}}{r_{0}{^2}}\right)= 1*14{^2}r_{0}{^2}\\
    \frac{14{^2}r_{0}{^2}x{^2}}{14{^2}} +\frac{r_{0}{^2}14{^2}y{^2}}{r_{0}{^2}}=14{^2}r_{0}{^2}\\
    r_{0}x{^2}+14{^2}y{^2}=14{^2}r_{0}{^2}\\
    14{^2}y{^2}=14{^2}r_{0}{^2}-r_{0}x{^2}\\
    y{^2}=\frac{14{^2}r_{0}{^2}-r_{0}x{^2}}{14{^2}}\\
    y{^2}=\frac{r_{0}{^2}}{14{^2}}(14{^2}-x{^2})\\
    y=\pm\sqrt{ \frac{r_{0}{^2}}{14{^2}}(14{^2}-x{^2}) }\\
    y=\frac{r_{0}}{14}*\pm\sqrt{(14{^2}-x{^2})}\\
    y=\frac{r_{0}}{14}*\pm\sqrt{196-x{^2}}
    \end{gather*}
    
    Sustituyendo $r_0=\frac{53}{2\pi}$ tenemos que $$\frac{r_{0}}{14}=\frac{\frac{53}{2\pi}}{14}\approx 0.6025\therefore y(x)=0.6025\pm \sqrt{ 196-x{^2} }$$
    Podemos observar que $x{^2}$ solo puede valor, máximo 196 ya que la función no está definida para la raíz cuadrada de un número negativo. $\therefore -14\leq x\leq 14$ 
    
    d) Muestre que la aproximación del volumen (7) con los 10 cilindros circunscritos de
    grosor $\Delta x = 1.4 cm$ y radios $y (x_{k})$ cm para$k = 0, 1, 2, . . . , 9$ es:
    $$V \approx 2 \times 2237.4593 = 4474.9186 cm3$$
    
    Partimos de que 
    $$V_{total}=\sum_{k=0}^{9}V_{k}$$
    donde $V_{k}=$ a cada volumen de disco con índice k y el volumen se traduce como$$V_{k}=\pi*\Delta h*r_{k}{^2}$$
    y el la medida del radio $r_{k}$ para cada cilindro circunscrito es el valor de $y(x_k)$, para $x_{k}=\Delta h*k$
    y $\Delta h$ es la altura correspondiente que es constante en cada cilindro.
    
    Así denotamos que \begin{gather*}
    \sum_{k=0}^{9}V_{k}=\sum_{k=0}^{9}\pi* r_{k}{^2}*\Delta h
    \end{gather*}
    donde $\pi$ y $\Delta h$ son siempre constantes, factorizamos
    \begin{gather*}
    \sum_{k=0}^{9}\pi* r_{k}{^2}*\Delta h
    =\pi \Delta h\sum_{k=0}^{9} r_{k}{^2}\\
    \text{sustituyendo }r_k \text{ por } y(x_{k})\\
    V_{tottal}=\pi \Delta h\sum_{k=0}^{9} y(x_{k}){^2}\\
    =\pi \Delta h\sum_{k=0}^{9} (0.6025 \sqrt{ 196-x_{k}{^2} }){^2}\\
    \end{gather*}
    así conseguimos la suma de los triangulo circunscritos para calcular el Balón de futbol.

    \begin{table}[!hbt]
        \begin{center}
        \begin{tabular}{| c | c | c | c | c | }
        \hline
        K & $\Delta h \cdot j$ & $y(x_{k})$ & $y(x_{k})^{2}$ & $\Delta h*\pi*y(x_{k})^2$ \\ \hline
        0 & 0    & 8.435       & 71.149225  & 312.930636 \\
        1 & 1.4  & 8.39271903  & 70.4377328 & 309.801329  \\
        2 & 2.8  & 8.26457839  & 68.303256  & 300.41341   \\
        3 & 4.2  & 8.04647716  & 64.7457948 & 284.766878  \\
        4 & 5.6  & 7.7308052   & 59.765349  & 262.861734  \\
        5 & 7    & 7.30492428  & 53.3619188 & 234.697977 \\
        6 & 8.4  & 6.748       & 45.535504  & 200.275607  \\
        7 & 9.8  & 6.02379488  & 36.2861048 & 159.594624   \\
        8 & 11.2 & 5.061       & 25.613721  & 112.655029   \\
        9 & 12.6 & 3.67673126  & 13.5183528 & 59.4568208   \\ \hline
    \multicolumn{5}{ |r| } {suma $2237.45404\; cm^3$}\\ \hline
        \end{tabular}
        \caption{Tabla de suma de los factores $x_k$}
        \label{tab:la suma de los cilindros circunscritos interpretados como una elipse}
        \end{center}
        \end{table}
        Como calculamos la mitad del balón, la suma de la parte derecha la multiplicamos por 2
$$\therefore V \approx 2 \times 2237.4540 = 4474.90809 \;cm3$$
a) Muestre que la aproximación del volumen (7) con los 10 cilindros inscritos de
grosor $\Delta x = 1.4 cm$ y radios $y (xk)$ cm para$k = 1, 2, . . . , 10$ es:
$$V \approx 2 \Delta 1924.5279 = 3849.0558 cm^3$$
Tenemos  la misma primicia del anterior.
$$V\approx\pi \Delta h\sum_{k=1}^{10} (0.6025 \sqrt{ 196-x_{k}{^2} }){^2}$$

\begin{table}[!hbt]
    \begin{center}
    \begin{tabular}{| c | c | c | c | c | }
    \hline
    K & $\Delta h \cdot j$ & $y(x_{k})$ & $y(x_{k})^{2}$ & $\Delta h*\pi*y(x_{k})^2$ \\ \hline
    1 & 1.4  &  8.39271903 & 70.4377328 & 309.801329 \\
    2 & 2.8  &  8.26457839 & 68.303256  & 300.41341 \\
    3 & 4.2  &  8.04647716 & 64.7457948 & 284.766878 \\
    4 & 5.6  &  7.7308052  & 59.765349  & 262.861734 \\
    5 & 7    &  7.30492428 & 53.3619188 & 234.697977 \\
    6 & 8.4  &  6.748      & 45.535504  & 200.275607 \\
    7 & 9.8  &  6.02379488 & 36.2861048 & 159.594624 \\
    8 & 11.2 &  5.061      & 25.613721  & 112.655029 \\
    9 & 12.6 &  3.67673126 & 13.5183528 & 59.4568208 \\
   10 & 14   & 0           & 0          & 0          \\ \hline
\multicolumn{5}{ |r| } {suma $1924.52341\; cm^3$}\\\hline
    \end{tabular}
    \caption{Tabla de suma de los factores $x_k$}
    \label{tab:la suma de los cilindros inscritos interpretados como una elipse}
    \end{center}
    \end{table}
    \break
    Como calculamos la mitad del balón, la suma de la parte derecha la multiplicamos por 2.
    $$\therefore V \approx 2 \times 1924.5279 = 3849.04682 cm3$$

    \section*{Un jugador de baseball}
    Un jugador de baseball batea la pelota a 3 ft de altura sobre el plato en dirección a la barda del jardín central que está a 400 ft de distancia de home y tiene 10 ft de altura. La pelota sale con rapidez de 115 $\frac{ft}{s}$ y un ángulo de elevación de 50° sobre la horizontal.
    
    1) Sabemos que el vector de movimiento $\vec{v}=115\frac{ft}{s}$ de la pelota se puede descomponer en dos direcciones de velocidad, movimientorectilineo y tiro veritcal y caida libre ya que su comprotamiento es en dos grados de libertad $$r:\left[0,t\right]\rightarrow \mathbb{R}^2_{(x,y)}$$
    $\therefore$ la componente horizontal del  vector de la bola está representada por la formula del mov. rect. uniforme.$x(t)=v_x(t)=v_{0_x}\Rightarrow x(t)-x_{0}=v_{0_x}(t)\therefore x(t)=v_{0_x}(t)$
    Como el movimiento no es recto completamente, si no que tiene un ángulo de inclinación, requerimos del coseno para calcular la fracció de la velocidad que le corresponde. $$\vec{\|vx\|}=\vec{\|v\|}cos\;\theta = 115cos\;50 = 73.91\frac{ft}{s}$$
    2) La componente vertical del vector resultante está representada por la caida libre, la cual denotamos como $v(t)=gt$.
    El área bajo la recta de velocidad respecto al tiempo representa el desplazamiento del objeto $$\therefore y(t)-y(0)=\frac{1}{2}gt^2$$
    como el movimiento de caida crece un lapso de tiempo y luego decrece hasta tocar el suelo nuestra aceleración es negativa y, además, consideremos que el movimiento empieza a una distancia distinda de cero.$$y(t)-y(0)=v_{0_y}t-\frac{1}{2}gt^2$$ tomando en cuenta que la gravedad es de $32\frac{ft}{s^2}$ $$\Rightarrow y(t)=y(0)+v_{0_y}t-\frac{1}{2}gt^2$$ donde $v_{0_y}$ represeta la velocidad respecto de la componente en y, $$\therefore \vec{\|v_y\|}=115 sen\theta 50=89.09\frac{ft}{s}$$, sustituyendo tenemos$$y(t)=3+89.09\frac{ft}{s}t-16t^2$$ donde 3 es la altura en pies donde el bat impacta con la pelota.
    
    \vspace{5mm} %5mm vertical space        
    
    3) Sabemos que $x(t)$ es la fŕomula de la distancia que tomará la bola respecto a cada segundo que dura el movimiento. Queremos averiguar cual sería el tiempo en el que la pelota está a 400 metros del bateador para poder determinar si la pelota sobrepasa la barrera
    $$x(t)=73.91\frac{ft}{s}t=400ft\therefore t=\frac{400ft}{73.91\frac{ft}{s}}=5.4119s$$
    La pelota tarda $5.4119s$ en llegar a los $400 ft$ de distancia.
    Ahora, para saber la posición de la pelota en ese tiempo tenemos que sustituir el valor del tiempo en la fórmula que nos dice a que altura llega la bola respecto a su desplazamiento.
    $$y(5.4119)=3+89.09\frac{ft}{s}(5.4119s)-16(5.4119s)^2$$
        $$y(5.4119) \approx 11.1156ft$$
    Concluimos que la pelota supera la barrera de $10ft$ de altura que está a $400 ft$ de distancia ya que la pelota en ese mismo instante se encuentra a una altura superior que el de la barrera.
    
    \vspace{5mm} %5mm vertical space        
    
    Consideremos que la bola choca con un obstaculo a $5ft$ de altura despues de pasar la barrera, lo primero que queremos saber es cuando $y(t)=5$
    \begin{gather*}
        y(t)=5=y(t)=3+89.09\frac{ft}{s}t-16t^2\\
        2=88.0951t-16t^2\\
        t=(88.0951-16t)t\\
        t_1=2\\
        t_2:(88.0951-16t_2)=2\\
        -16t_2=-86.0951\\
        t_2=\frac{-86.0951}{-16}
        t_2=5.4839s
    \end{gather*}
    Tengamos en cuenta el movimeinto parabólico que tiene la pelota, entonces pasa en 2 tiempos diferentes a los $5 fts$, considerando el mayor de ellos (parte final del movimiento), tenemos que, para saber la posición a la que se encntraba la bola en ese tiempo usaremos la formula de posición $x(t)$.
    \begin{gather*}
        x(t)=73.9206t\\
        x(5.4839)=73.9206\frac{ft}{s}(5.4839s)\\
        x(5.4839)\approx405.314ft
    \end{gather*}
    $\therefore$ Concluimos que el movimiento termimina, o por lo menos no está definido despues de $5.4839s$ en la distancia max. de $405.314ft$.
    
    \vspace{5mm} %5mm vertical space        
    
    c) Para calcular el valor máximo de altura a la que se encuentra la bola en el movimiento tenemos que hacer uso de la primer derivada de la fórmula $y(t)$ que representa la posición de altura para cada valor del tiempo.
    \begin{gather*}
        y'(t)=88.0951t-32t\\
        y'(t)=0=88.0951t-32t\\
        -32t=-88.0951\\
        t=\frac{-88.0951}{-32}\\
        t\approx 2.7529s
    \end{gather*}
    $t\approx 2.7529s$ es el valor aproximado del tiempo en el que se encuentra a la altura máxima, este valor lo sustituimos en la ecuación $y(t)$ para saber cual es la posición en ese tiempo.
    \begin{gather*}
        y(2.7529s)=3+89.09\frac{ft}{s}(2.7529s)-16(2.7529s)^2
        y(2.7529s)\approx 124.2617ft
    \end{gather*}
\end{document}

\chapter*{Ejercicio 3}
\section*{Convergencia de la Serie Geométrica}

    **Problema:**
    Demostrar que la serie geométrica $\sum_{k=1}^{\infty} q^k$ diverge a $+\infty$ cuando $q \geq 1$.

    **Solución:**

    \textbf{Serie geométrica general:}
    \[
    \sum_{k=1}^{\infty} q^k
    \]
    Esta es una serie geométrica con razón \( q \).
 
    \textbf{Condiciones de convergencia de una serie geométrica:}
    Para que la serie geométrica \(\sum_{k=1}^{\infty} q^k\) converja, es necesario que \( |q| < 1 \). Si \( |q| \geq 1 \), la serie diverge. Esto se debe a que:
    \begin{itemize}
    \item Si \( |q| < 1 \), los términos \( q^k \) tienden a cero conforme \( k \) aumenta, y la suma total se aproxima a un valor finito.
    \item Si \( |q| \geq 1 \), los términos \( q^k \) no tienden a cero, lo que provoca que la serie no alcance un valor finito y, por lo tanto, diverge.
    \end{itemize}
 
    \textbf{Caso \( q \geq 1 \):}
    \begin{itemize}
    \item Si \( q = 1 \), la serie se convierte en:
      \[
      \sum_{k=1}^{\infty} 1^k = \sum_{k=1}^{\infty} 1 = 1 + 1 + 1 + \cdots
      \]
      Claramente, esta serie suma infinitamente muchos unos, lo que implica que la serie diverge a \( +\infty \).
    \item Si \( q > 1 \), los términos de la serie \( q^k \) crecen sin límite conforme \( k \) aumenta. Por lo tanto, la serie se comporta como:
      \[
      \sum_{k=1}^{\infty} q^k = q + q^2 + q^3 + \cdots
      \]
      Cada término es mayor que el anterior, lo que hace que la suma total también diverge a \( +\infty \).
    \end{itemize}
 
 \textbf{Conclusión:}
 - Por lo tanto, si \( q \geq 1 \), la serie geométrica \(\sum_{k=1}^{\infty} q^k\) no converge, y en términos más precisos, diverge a \( +\infty \).
 
 \section*{b) Valores de \( q \) para los cuales la serie \(\sum_{k=1}^{\infty} q^k\) converge}
 
 \textbf{Análisis:}
 
 1. \textbf{Caso \( q \leq 0 \):}
    \begin{itemize}
    \item Para \( q \leq 0 \), el comportamiento de la serie depende del valor absoluto de \( q \):
      \begin{itemize}
      \item Si \( -1 < q < 0 \), la serie es geométrica y converge porque \( |q| < 1 \). Aquí, los términos \( q^k \) se alternan en signo y disminuyen en magnitud.
      \item Si \( q = -1 \), la serie se convierte en:
        \[
        \sum_{k=1}^{\infty} (-1)^k
        \]
        Esta es una serie alternante cuyos términos no tienden a cero, por lo tanto, no converge (es un caso de divergencia condicional).
      \item Si \( q \leq -1 \), los términos \( q^k \) oscilan en magnitud (y posiblemente divergen si \( q < -1 \)) o se repiten (en el caso de \( q = -1 \)), lo que impide la convergencia.
      \end{itemize}
    \end{itemize}
 
 \textbf{Conclusión:}
 - \textbf{La serie geométrica \(\sum_{k=1}^{\infty} q^k\) converge si y solo si \( -1 < q < 1 \).}